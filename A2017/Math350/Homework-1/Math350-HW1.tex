\documentclass{article}

\usepackage[utf8]{inputenc}
\usepackage{comment}
\usepackage[french]{isodate}

\usepackage{graphicx}
\usepackage{siunitx}
\usepackage{paracol}
\usepackage{amsmath}
\usepackage{ amssymb }
\usepackage[utf8]{inputenc}
\usepackage[bookmarks=true]{hyperref}
\usepackage{bookmark}

\usepackage{mathtools,xparse}

\DeclarePairedDelimiter{\abs}{\lvert}{\rvert}
\DeclarePairedDelimiter{\norm}{\lVert}{\rVert}

%Math typeset and settings
\sisetup{output-decimal-marker = {,}}
\newcommand*{\ft}[1]{_\mathrm{#1}} 
\newcommand*{\dd}{\mathop{}\!\mathrm{d}}
\newcommand*{\tran}{^{\mkern-1.5mu\mathsf{T}}}%transpose of matrix


%Math shortcuts
\newcommand{\vout}{v\ft{out}}

\begin{document}
	\begin{titlepage}
		\begin{center}
			\vspace*{1cm}
			\textbf{Math350}\\
			\text{Graph theory}\\
			\vspace{0.5cm}
			Homework I
			
			\vspace{1.5cm}
			
			\textbf{Frédéric Boileau}\\
			\vspace{2cm}
			Prof. 
			Jan Volec
			\vfill
			\today
			\thispagestyle{empty}
		\end{center}
	\end{titlepage}
	\newpage
	\pdfbookmark{\contentsname}{Contents}
	\tableofcontents
	\thispagestyle{empty}
	\clearpage
	
	\section{}
	a) True\\
	Let G be a graph with $\vert V(G) \vert \ge 2 $. We want to prove that there exists at least one pair of vertices in G that have the same degree. We argue by contradiction. Assume not, then all the vertices have distinct degree.
	\begin{align*}
		\text{Let } k= \vert V(G) \vert
	\end{align*}
	Then if all the vertices have different degree we can enumerate the degrees as follows:
	\begin{align*}
		v_i \in V(G) \text{ with } \deg(v_i) = i \text{ and } 0 \leq i \leq n-1
	\end{align*}
	But if  $\deg v_{n-1} = n-1$ then it is connected to all the other edges, so there cannot be an edge with degree zero. $\Rightarrow \Leftarrow$\\[2ex]
	b) False\\
	Consider the graph consisting of two triangles ${u,u_1,v}$ and ${w,w_1,v}$ where the edges thus described are the only ones.\\[2ex]
	c) True\\
	Let $C_1$ be the cycle containing e and f and $C_2$ be the one containing f and g. We will build the cycle containing e and g. Let $\{u_i\}$ and $\{v_i\}$ respectively denote the vertices of $C_1$ and $C_2$. Let $u_1$ be an endpoint of e and $u_r$ be the last vertex of $C_1$ not in $C_2$. Then $u_1...u_r$ is obviously a path. For the next vertex we get on $C_2\backslash C_1 $. Extend the path until we cross the edge g and then continue until we get a vertex, say $v_l$, which is the last in $C_2$ but not in $C_1$. We then get on $C_1 \backslash C_2$ and continue until we get to the other endpoint of e. 
	
	Notes : 
	
	The second part of the cycle (when get back on $C_1$) arrives at the other endpoint of e because we have skipped f, the most simple case being when $V(C_1) \cap V(C_2)$ is the endpoints of f. 
	
	Also we are able to extend the path on $C_2\backslash C_1$ and $C_1 \backslash C_2$ to go through g because otherwise g would be in $C_1$ already.
	\\[2ex]
	d)True\\
	Consider $T-v$ with $\deg v = k$. Now we have split the tree into k components. Call them $T_1,...T_k$. If some of them have only one vertex, let there be l of them. Consider now the $T_{l+1},...,T_{k}$ trees that have more than one vertex. Since they are trees they have at least two leaves. One of those leaves is therefore not $v$. Take one of those leaves and per tree $T_{l+1},...,T_{k}$. Let $w$ denote any of them. It is easy to see that
	\begin{equation}
		\deg_{T_i} w = \deg_T w \quad \forall w
	\end{equation}
	So they are leaves in T as well. So for we have $k-l$ leaves. Now consider the trees $T_k$ with only one vertex, this implies that their vertex was a leaf in T. There are l of them. Add the two set of leaves and we get k leaves.\\
	\clearpage
	\section{}
	G is a graph that is not connected. Let it have k components. \\
	Let $v,w \in G$
	\begin{align*}
		vw \notin E(G) &\Rightarrow vw \in E(\bar{G})\\
		vw \in E(G) &\Rightarrow v,w \in V(F) \subset V(G) 
	\end{align*}
	With F being a component of G.\\
	\begin{align*}
		&\exists \, u \quad s.t. \quad uv ,uw \notin E(G)\\
		&uv,uw \in E(\bar{G}) \\
		&\therefore u \sim_w w \text{ in } \bar{G} 
	\end{align*}
	\section{}
	
	Let C be the cycle of minimal length in G. We argue by contradiction. Assume it is longer than $2k+1$. Let $\{v_1,...,v_{r}\}$ be the ordered list of the vertices in C and let P denote the path of length k between $v_1$ and $v_{k+1}$. Since the cycle has length at least $2k+2$ then the path from $v_{k+1}$ back to $v_1$ is at least $k+2$. Which means that the path from $v_{k+1}$ to $v_r$ is at least $k+1$. But there should exist a path from $v_{k+1}$ to $v_{r}$ of length k disjoint from P by assumption. So the cycle was not minimal hence we derive a contradiction.
	
	\section{}
	
	Let $P_1$ and $P_2$ denote two paths in G of length k. Express their edges as $P_1 = {u_1,...,u_r}$ and $P_2 = v_1,...,v_r$ with $r= k+1$. We argue by contradiction and so assume that they are disjoint. Since G is connected there exists a path from $v_1$ to $u_i$ for some $i$. Let Q denote such a path. Let $v_j$ be the last vertex in Q and in $P_2$. $P_2$ is a path so $v_1,...,v_j \subset P_2$ is a subpath of Q. 
	\begin{align*}
		Q = v_1,...,v_j,\{\zeta\},u_i 
	\end{align*}
	Where $\{\zeta\}$ is the set of vertices outside both $P_1$ and $P_2$. \\
	i) \begin{align*}
		\zeta_r \notin P_1,P_2 \quad \forall \quad r\in \mathbb{N}\\
		i<j \Rightarrow v_1,...,v_j,\{\zeta\},u_i,...,u_k \text{ is a path with length more than k}\\
		j<i \Rightarrow v_1,...,v_j,\{\zeta\},u_i,...,u_1 \text{ is a path with length more than k}\\
	\end{align*}
	ii) $\nexists$ $\zeta \notin P_1 \Rightarrow$, then consider the preceding argument without the zetas and we still obtain a path with length larger than k. 
	
	\clearpage
	
	
	\section{}
	
	We use induction, the base case is trivial for order 3 since we have only two possible connected subgraphs that are not just a vertex. Consider the tree with order 4, we get at least 2 leaves and it is easy to see that if we choose a set of subgraphs that have a not empty intersection pairwise, then the intersection of all of them is not empty.
	For induction  now assume that for a all trees T' of order n-1 whenever
	\begin{align}
		V(T'_i) \bigcap V(T'_j) \neq \emptyset \quad \forall \quad  1 \leq i,j \leq k \qquad \Rightarrow \quad  \bigcap_{i=1}^{k}V(T'_i)\neq \emptyset
	\end{align}
	For all $T'_i$ that are connected subgraphs of T'. \\
	Now consider the tree T of order n with 
	\begin{align}
		V(T_i) \bigcap V(T_j) \neq \emptyset \quad \forall \quad  1 \leq i,j \leq k
	\end{align}
	For all $T_i$ that are connected subgraphs of T. 
	Let v be a leaf of T. Then obviously (3) still holds for T- v:
	
	In a tree every vertex is connected to another via a unique path. Therefore if a leaf is in the intersection of two connected subgraphs, then the vertex adjacent to it, say u, is also in the intersection. 
	
	Well then by the induction hypothesis we have that (2) holds for T-v.  As we have just seen if v is in the intersection of two trees then so is u and so adding v will not affect the property stated in (2).
	
	
	
	
	
	
\end{document}