\documentclass{article}

\usepackage{comment}
\usepackage[french]{isodate}

\usepackage{graphicx}
\usepackage{siunitx}
\usepackage{paracol}
\usepackage{amsmath}
\usepackage{ amssymb }
\usepackage[utf8]{inputenc}
\usepackage[bookmarks=true]{hyperref}
\usepackage{bookmark}

\usepackage{mathtools,xparse}

\DeclarePairedDelimiter{\abs}{\lvert}{\rvert}
\DeclarePairedDelimiter{\norm}{\lVert}{\rVert}

%Math typeset and settings
\sisetup{output-decimal-marker = {,}}
\newcommand*{\ft}[1]{_\mathrm{#1}} 
\newcommand*{\dd}{\mathop{}\!\mathrm{d}}
\newcommand*{\tran}{^{\mkern-1.5mu\mathsf{T}}}%transpose of matrix

\setlength{\jot}{10pt}



\begin{document}
	\begin{titlepage}
		\begin{center}
			\vspace*{1cm}
			\textbf{Math 316}\\
			\text{Complex Variables}\\
			\vspace{0.5cm}
			Homework I
			
			\vspace{1.5cm}
			
			\textbf{Frédéric Boileau}\\
			\vspace{2cm}
			Prof. 
			John Toth
			\vfill
			\today
			\thispagestyle{empty}
		\end{center}
	\end{titlepage}
	\newpage
	\pdfbookmark{\contentsname}{Contents}
	\tableofcontents
	\thispagestyle{empty}
	\clearpage
	
	\section{}
	\subsection{}
	To prove that a function is holomorphic we keep in mind that products and sums, products and quotients of holomorphic functions are construct holomorphic functions (provided that there is no division by zero)\\[2ex]
	a)
		\begin{equation*}
		f(z) = \sin(z) - \frac{z^2}{z+1}
	\end{equation*}
	$\sin(z)$ is a trigonometric function so it is holomorphic, $z^2$ and $z+1$ are both polynomials and so are holomorphic, therefore their quotient is holomorphic therefore the sum of the two parts is holomorphic when $z \neq -1$\\[2ex]
	c) 
		\begin{equation*}
			h(z) = \frac{\cos(z)}{z^2 + 1}
		\end{equation*}
	$\cos(z)$ is a trigonometric function so it is homomorphic, $z^2 +1 $ is a polynomial is z so it is holomorphic. Their quotient is therefore holomorphic when $z \neq i ,-i$\\[2ex]
	\subsection{}
	
		\begin{align*}
			g(z) &= \frac{\bar{z}}{z^2+1} = \frac{x-iy}{x^2 -i2xy + 2} = \frac{(x-iy)(x^2+2 +i2xy)}{(x^2+2 )^2 + 4x^2y^2}\\
			u(x,y) &= \frac{x^3 +2x + 2xy^2}{(x^2+2 )^2 + 4x^2y^2}\\
			v(x,y) &= \frac{-x^2y -2y +2x^2y}{(x^2+2 )^2 + 4x^2y^2}
		\end{align*}
		Both $u(x,y)$ and $v(x,y)$ have the same denominator so let's just examine their numerator, say $n_u(x,y)$ and $n_v(x,y)$
		\begin{align*}
			\frac{\partial n_u(x,y)}{den}
		\end{align*}
	
\clearpage
\section{dump}
	
	
	
	
	
\end{document}